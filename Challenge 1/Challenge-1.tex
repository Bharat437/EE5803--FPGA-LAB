\documentclass{article}
\usepackage{setspace}
\usepackage{gensymb}

\singlespacing

\usepackage[cmex10]{amsmath}

\usepackage{amsthm}
\usepackage[latin1]{inputenc}
\usepackage{mathrsfs}
\usepackage{txfonts}
\usepackage{stfloats}
\usepackage{bm}
\usepackage{cite}
\usepackage{cases}
\usepackage{subfig}
\usepackage{karnaugh-map}
\usepackage{longtable}
\usepackage{multirow}

\usepackage{enumitem}
\usepackage{mathtools}
\usepackage{steinmetz}
\usepackage{tikz}
\usepackage{circuitikz}
\usepackage{verbatim}
\usepackage{tfrupee}
\usepackage[breaklinks=true]{hyperref}

\usepackage{tkz-euclide}

\usetikzlibrary{arrows, shapes.gates.logic.US, calc}
\usepackage{listings}
    \usepackage{color}                                            %%
    \usepackage{array}                                            %%
    \usepackage{longtable}                                        %%
    \usepackage{calc}                                             %%
    \usepackage{multirow}                                         %%
    \usepackage{hhline}                                           %%
    \usepackage{ifthen}                                           %%
    \usepackage{lscape}     
\usepackage{multicol}
\usepackage{chngcntr}

\DeclareMathOperator*{\Res}{Res}

\renewcommand\thesection{\arabic{section}}
\renewcommand\thesubsection{\thesection.\arabic{subsection}}
\renewcommand\thesubsubsection{\thesubsection.\arabic{subsubsection}}

\renewcommand\thesectiondis{\arabic{section}}
\renewcommand\thesubsectiondis{\thesectiondis.\arabic{subsection}}
\renewcommand\thesubsubsectiondis{\thesubsectiondis.\arabic{subsubsection}}


\hyphenation{op-tical net-works semi-conduc-tor}
\def\inputGnumericTable{}                                 %%

\lstset{
%language=C,
frame=single, 
breaklines=true,
columns=fullflexible
}
\AtBeginDocument{\hypersetup{pdfborder={0 0 0}}}
\begin{document}


\newtheorem{theorem}{Theorem}[section]
\newtheorem{problem}{Problem}
\newtheorem{proposition}{Proposition}[section]
\newtheorem{lemma}{Lemma}[section]
\newtheorem{corollary}[theorem]{Corollary}
\newtheorem{example}{Example}[section]
\newtheorem{definition}[problem]{Definition}

\newcommand{\BEQA}{\begin{eqnarray}}
\newcommand{\EEQA}{\end{eqnarray}}
\newcommand{\define}{\stackrel{\triangle}{=}}
\bibliographystyle{IEEEtran}
\providecommand{\mbf}{\mathbf}
\providecommand{\pr}[1]{\ensuremath{\Pr\left(#1\right)}}
\providecommand{\qfunc}[1]{\ensuremath{Q\left(#1\right)}}
\providecommand{\sbrak}[1]{\ensuremath{{}\left[#1\right]}}
\providecommand{\lsbrak}[1]{\ensuremath{{}\left[#1\right.}}
\providecommand{\rsbrak}[1]{\ensuremath{{}\left.#1\right]}}
\providecommand{\brak}[1]{\ensuremath{\left(#1\right)}}
\providecommand{\lbrak}[1]{\ensuremath{\left(#1\right.}}
\providecommand{\rbrak}[1]{\ensuremath{\left.#1\right)}}
\providecommand{\cbrak}[1]{\ensuremath{\left\{#1\right\}}}
\providecommand{\lcbrak}[1]{\ensuremath{\left\{#1\right.}}
\providecommand{\rcbrak}[1]{\ensuremath{\left.#1\right\}}}
\theoremstyle{remark}
\newtheorem{rem}{Remark}
\newcommand{\sgn}{\mathop{\mathrm{sgn}}}
%\providecommand{\abs}[1]{\left\vert#1\right\vert}
\providecommand{\res}[1]{\Res\displaylimits_{#1}} 
%\providecommand{\norm}[1]{\left\lVert#1\right\rVert}
%\providecommand{\norm}[1]{\lVert#1\rVert}
\providecommand{\mtx}[1]{\mathbf{#1}}
%\providecommand{\mean}[1]{E\left[ #1 \right]}
\providecommand{\fourier}{\overset{\mathcal{F}}{ \rightleftharpoons}}
%\providecommand{\hilbert}{\overset{\mathcal{H}}{ \rightleftharpoons}}
\providecommand{\system}{\overset{\mathcal{H}}{ \longleftrightarrow}}
	%\newcommand{\solution}[2]{\textbf{Solution:}{#1}}
\newcommand{\solution}{\noindent \textbf{Solution: }}
\newcommand{\cosec}{\,\text{cosec}\,}
\providecommand{\dec}[2]{\ensuremath{\overset{#1}{\underset{#2}{\gtrless}}}}
\newcommand{\myvec}[1]{\ensuremath{\begin{pmatrix}#1\end{pmatrix}}}
\newcommand{\mydet}[1]{\ensuremath{\begin{vmatrix}#1\end{vmatrix}}}
\numberwithin{equation}{subsection}
\makeatletter
\@addtoreset{figure}{problem}
\makeatother
\let\StandardTheFigure\thefigure
\let\vec\mathbf
\renewcommand{\thefigure}{\theproblem}
\def\putbox#1#2#3{\makebox[0in][l]{\makebox[#1][l]{}\raisebox{\baselineskip}[0in][0in]{\raisebox{#2}[0in][0in]{#3}}}}
     \def\rightbox#1{\makebox[0in][r]{#1}}
     \def\centbox#1{\makebox[0in]{#1}}
     \def\topbox#1{\raisebox{-\baselineskip}[0in][0in]{#1}}
     \def\midbox#1{\raisebox{-0.5\baselineskip}[0in][0in]{#1}}
\vspace{3cm}
\title{Assignment 1}
\author{AVVARU BHARAT}
\date{}
\maketitle
\renewcommand\thefigure{\arabic{figure}}
\setcounter{figure}{0}
\renewcommand{\thetable}{\arabic{table}}
\setcounter{table}{0}
Download all latex-tikz codes from 
%
\begin{lstlisting}
https://github.com/Bharat437/EE5803-FPGA-LAB/tree/main/Assignment-1
\end{lstlisting}
%
\section{Problem}
Obtain and implement an algorithm to convert any truth table to NAND logic.
\section{Algorithm}
1. Obtain K-Map for given truth table.\\
2. From K-Map obtain NAND Logic instead of SOP form.
\section{Explanation}
For example the given truth table is as below\\
\begin{table} [h!]
    \centering
    \begin{tabular}{ | c | c | c | c | }
    \hline
    X & Y & Z & G(X,Y,Z) \\
    \hline
    0 & 0 & 0 & 0 \\
    0 & 0 & 1 & 0 \\
    0 & 1 & 0 & 1 \\
    0 & 1 & 1 & 0 \\
    1 & 0 & 0 & 1 \\
    1 & 0 & 1 & 1 \\
    1 & 1 & 0 & 0 \\
    1 & 1 & 1 & 1 \\
     \hline
\end{tabular}
\caption{Given Truth table}
\label{Table1}
\end{table}
\newline
Now the K-Map for given truth table \ref{Table1} is as below.\\
\begin{figure}[h!]
\centering
{
    \begin{karnaugh-map}[4][2][1][][]
        
        \minterms{2,4,5,7}
        \maxterms{0,1,3,6}
        \implicant{4}{5}
        \implicant{2}{2}
        \implicant{7}{7}
        \draw[color=black, ultra thin] (0, 2) --
        node [pos=0.7, above right, anchor=south west] {$YZ$} % Y label
        node [pos=0.7, below left, anchor=north east] {$X$} % X label
        ++(135:1);
    \end{karnaugh-map}
}
\caption{K-Map for given truth table}
\label{kmap}
\end{figure}
\\\\
Now the NAND logic from K-Map in figure \ref{kmap} is as follows.
\begin{equation}
\begin{split}
    G=\overline{\left(\overline{X.\overline{Y}}\right).\left(\overline{X.Y.Z}\right).\left(\overline{\overline{X}.Y.\overline{Z}}\right)}
\end{split}
\end{equation}
Now we will draw logic circuit according to the above expression.

\begin{figure}[h!]
\begin{center}
\begin{tikzpicture}[ circuit symbol wires, connection/.style={draw,circle,fill=black,inner sep=1.5pt}]
    \node (x) at (0,4.5) {$X$};
    \node (y) at (0,3) {$Y$};
    \node (z) at (0,1.5) {$Z$};
    
    \node[nand gate US, minimum size=32pt, draw, logic gate inputs=ni] at ($(x) + (3, -0.187)$) (nand1) {};
    \node[nand gate US, minimum size=32pt, draw, logic gate inputs=nnn] at ($(y) + (3, 0)$) (nand2) {};
    \node[nand gate US, minimum size=32pt, draw, logic gate inputs=ini] at ($(z) + (3, 0.272)$) (nand3) {};
    
    \draw (x.east) -- ++(right:3mm) |- (nand1.input 1);
    \draw (y.east) - ++(right:1cm) node[connection,pos=1] |- (nand1.input 2) ;
    \draw (x.east) -- ++(right:3mm) |- (nand2.input 1) node[connection,pos=0.5];
    \draw (y.east) - ++(right:3mm) |- (nand2.input 2);
    \draw (z.east) - ++(right:1.5cm) node[connection,pos=1] |- (nand2.input 3);
    \draw (x.east) -- ++(right:3mm) |- (nand3.input 1);
    \draw (y.east) - ++(right:1cm) node[connection,pos=1] |- (nand3.input 2) ;
    \draw (z.east) -- (z.east) |- (nand3.input 3);
    
    \node[nand gate US, minimum size=32pt, draw, logic gate inputs=nnn] at ($(y) + (6, 0)$) (nand4) {};
    
    \draw (nand1.output) -- ([xshift=0.5cm]nand1.output) |- (nand4.input 1);
    \draw (nand2.output) -- (nand2.output) |- (nand4.input 2);
    \draw (nand3.output) -- ([xshift=0.5cm]nand3.output) |- (nand4.input 3);
    
    \draw (nand4.output) -- node[above]{$G$} ($(nand4) + (2, 0)$);
\end{tikzpicture}
\end{center}
\caption{Logic circuit using NAND gate for given truth table}
\label{ckt1}
\end{figure}
Whereas SOP form from K-Map in figure \ref{kmap} is given as below.
\begin{equation}
\begin{split}
    G=X.\overline{Y}+X.Y.Z+\overline{X}.Y.\overline{Z}
\end{split}
\end{equation}
Now the logic circuit for above SOP form expression is as given below.\\\\\\\\\\\\

\begin{figure}[h!]
\begin{center}
\begin{tikzpicture}[ circuit symbol wires, connection/.style={draw,circle,fill=black,inner sep=1.5pt}]
    \node (x) at (0,4.5) {$X$};
    \node (y) at (0,3) {$Y$};
    \node (z) at (0,1.5) {$Z$};
    
    \node[and gate US, minimum size=32pt, draw, logic gate inputs=ni] at ($(x) + (3, -0.187)$) (nand1) {};
    \node[and gate US, minimum size=32pt, draw, logic gate inputs=nnn] at ($(y) + (3, 0)$) (nand2) {};
    \node[and gate US, minimum size=32pt, draw, logic gate inputs=ini] at ($(z) + (3, 0.272)$) (nand3) {};
    
    \draw (x.east) -- ++(right:3mm) |- (nand1.input 1);
    \draw (y.east) - ++(right:1cm) node[connection,pos=1] |- (nand1.input 2) ;
    \draw (x.east) -- ++(right:3mm) |- (nand2.input 1) node[connection,pos=0.5];
    \draw (y.east) - ++(right:3mm) |- (nand2.input 2);
    \draw (z.east) - ++(right:1.5cm) node[connection,pos=1] |- (nand2.input 3);
    \draw (x.east) -- ++(right:3mm) |- (nand3.input 1);
    \draw (y.east) - ++(right:1cm) node[connection,pos=1] |- (nand3.input 2) ;
    \draw (z.east) -- (z.east) |- (nand3.input 3);
    
    \node[or gate US, minimum size=32pt, draw, logic gate inputs=nnn] at ($(y) + (6, 0)$) (nand4) {};
    
    \draw (nand1.output) -- ([xshift=0.5cm]nand1.output) |- (nand4.input 1);
    \draw (nand2.output) -- (nand2.output) |- (nand4.input 2);
    \draw (nand3.output) -- ([xshift=0.5cm]nand3.output) |- (nand4.input 3);
    
    \draw (nand4.output) -- node[above]{$G$} ($(nand4) + (2, 0)$);
\end{tikzpicture}
\end{center}
\caption{Logic circuit from SOP form}
\label{ckt2}
\end{figure}
\section{Conclusion}
From the above circuit figures \ref{ckt1} and \ref{ckt2}, we can say that SOP form i.e. AND-OR logic is equivalent to NAND-NAND logic. The Verification is also done using a c code.
\end{document}
