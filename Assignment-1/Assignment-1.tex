\documentclass{article}
\usepackage{setspace}
\usepackage{gensymb}

\singlespacing

\usepackage[cmex10]{amsmath}

\usepackage{amsthm}
\usepackage[latin1]{inputenc}
\usepackage{mathrsfs}
\usepackage{txfonts}
\usepackage{stfloats}
\usepackage{bm}
\usepackage{cite}
\usepackage{cases}
\usepackage{subfig}

\usepackage{longtable}
\usepackage{multirow}

\usepackage{enumitem}
\usepackage{mathtools}
\usepackage{steinmetz}
\usepackage{tikz}
\usepackage{circuitikz}
\usepackage{verbatim}
\usepackage{tfrupee}
\usepackage[breaklinks=true]{hyperref}

\usepackage{tkz-euclide}

\usetikzlibrary{arrows, shapes.gates.logic.US, calc}
\usepackage{listings}
    \usepackage{color}                                            %%
    \usepackage{array}                                            %%
    \usepackage{longtable}                                        %%
    \usepackage{calc}                                             %%
    \usepackage{multirow}                                         %%
    \usepackage{hhline}                                           %%
    \usepackage{ifthen}                                           %%
    \usepackage{lscape}     
\usepackage{multicol}
\usepackage{chngcntr}

\DeclareMathOperator*{\Res}{Res}

\renewcommand\thesection{\arabic{section}}
\renewcommand\thesubsection{\thesection.\arabic{subsection}}
\renewcommand\thesubsubsection{\thesubsection.\arabic{subsubsection}}

\renewcommand\thesectiondis{\arabic{section}}
\renewcommand\thesubsectiondis{\thesectiondis.\arabic{subsection}}
\renewcommand\thesubsubsectiondis{\thesubsectiondis.\arabic{subsubsection}}


\hyphenation{op-tical net-works semi-conduc-tor}
\def\inputGnumericTable{}                                 %%

\lstset{
%language=C,
frame=single, 
breaklines=true,
columns=fullflexible
}
\begin{document}


\newtheorem{theorem}{Theorem}[section]
\newtheorem{problem}{Problem}
\newtheorem{proposition}{Proposition}[section]
\newtheorem{lemma}{Lemma}[section]
\newtheorem{corollary}[theorem]{Corollary}
\newtheorem{example}{Example}[section]
\newtheorem{definition}[problem]{Definition}

\newcommand{\BEQA}{\begin{eqnarray}}
\newcommand{\EEQA}{\end{eqnarray}}
\newcommand{\define}{\stackrel{\triangle}{=}}
\bibliographystyle{IEEEtran}
\providecommand{\mbf}{\mathbf}
\providecommand{\pr}[1]{\ensuremath{\Pr\left(#1\right)}}
\providecommand{\qfunc}[1]{\ensuremath{Q\left(#1\right)}}
\providecommand{\sbrak}[1]{\ensuremath{{}\left[#1\right]}}
\providecommand{\lsbrak}[1]{\ensuremath{{}\left[#1\right.}}
\providecommand{\rsbrak}[1]{\ensuremath{{}\left.#1\right]}}
\providecommand{\brak}[1]{\ensuremath{\left(#1\right)}}
\providecommand{\lbrak}[1]{\ensuremath{\left(#1\right.}}
\providecommand{\rbrak}[1]{\ensuremath{\left.#1\right)}}
\providecommand{\cbrak}[1]{\ensuremath{\left\{#1\right\}}}
\providecommand{\lcbrak}[1]{\ensuremath{\left\{#1\right.}}
\providecommand{\rcbrak}[1]{\ensuremath{\left.#1\right\}}}
\theoremstyle{remark}
\newtheorem{rem}{Remark}
\newcommand{\sgn}{\mathop{\mathrm{sgn}}}
%\providecommand{\abs}[1]{\left\vert#1\right\vert}
\providecommand{\res}[1]{\Res\displaylimits_{#1}} 
%\providecommand{\norm}[1]{\left\lVert#1\right\rVert}
%\providecommand{\norm}[1]{\lVert#1\rVert}
\providecommand{\mtx}[1]{\mathbf{#1}}
%\providecommand{\mean}[1]{E\left[ #1 \right]}
\providecommand{\fourier}{\overset{\mathcal{F}}{ \rightleftharpoons}}
%\providecommand{\hilbert}{\overset{\mathcal{H}}{ \rightleftharpoons}}
\providecommand{\system}{\overset{\mathcal{H}}{ \longleftrightarrow}}
	%\newcommand{\solution}[2]{\textbf{Solution:}{#1}}
\newcommand{\solution}{\noindent \textbf{Solution: }}
\newcommand{\cosec}{\,\text{cosec}\,}
\providecommand{\dec}[2]{\ensuremath{\overset{#1}{\underset{#2}{\gtrless}}}}
\newcommand{\myvec}[1]{\ensuremath{\begin{pmatrix}#1\end{pmatrix}}}
\newcommand{\mydet}[1]{\ensuremath{\begin{vmatrix}#1\end{vmatrix}}}
\numberwithin{equation}{subsection}
\makeatletter
\@addtoreset{figure}{problem}
\makeatother
\let\StandardTheFigure\thefigure
\let\vec\mathbf
\renewcommand{\thefigure}{\theproblem}
\def\putbox#1#2#3{\makebox[0in][l]{\makebox[#1][l]{}\raisebox{\baselineskip}[0in][0in]{\raisebox{#2}[0in][0in]{#3}}}}
     \def\rightbox#1{\makebox[0in][r]{#1}}
     \def\centbox#1{\makebox[0in]{#1}}
     \def\topbox#1{\raisebox{-\baselineskip}[0in][0in]{#1}}
     \def\midbox#1{\raisebox{-0.5\baselineskip}[0in][0in]{#1}}
\vspace{3cm}
\title{Assignment 1}
\author{AVVARU BHARAT}
\date{}
\maketitle
\renewcommand\thefigure{\arabic{figure}}
\setcounter{figure}{0}
Download all latex-tikz codes from 
%
\begin{lstlisting}
https://github.com/Bharat437/EE5803-FPGA-LAB/tree/main/Assignment-1
\end{lstlisting}
%
\section{Problem}
Draw the Logic Circuit for the following Boolean Expression:\\
$\left(\vec{X}^{'}+\vec{Y}\right).\vec{Z}+\vec{W}^{'}$
\section{Explanation}
First we will simplify the given Boolean expression as below, so that NAND or NOR logic gates are used for designing logic circuit.\\
Using De Morgan's Law
\begin{equation}
\begin{split}
    \left(\overline{X}+Y\right).Z+\overline{W}&=\left(\overline{X}+\overline{\overline{Y}}\right).Z+\overline{W}\\
    &=\overline{X.\overline{Y}}.Z+\overline{W}\\
    &=\left(\overline{\overline{\overline{X.\overline{Y}}.Z}}\right)+\overline{W}
\end{split}
\end{equation}
\begin{equation}
    \implies\left(\overline{X}+Y\right).Z+\overline{W}=\overline{\left(\overline{\overline{X.\overline{Y}}.Z}\right).W}
\end{equation}
Now we will draw logic circuit according to the above simplified expression.
\begin{figure}
\begin{center}
\begin{tikzpicture}[ circuit symbol wires]
    \node (x) at (0,4.5) {$X$};
    \node (y) at (0,3) {$Y$};
    \node (z) at (0,1.5) {$Z$};
    \node (w) at (0,0.5) {$W$};
    
    \node[nand gate US, minimum size=32pt, draw, logic gate inputs=nn] at ($(x) + (4, -0.17)$) (nand2) {};
    \node[nand gate US, minimum size=32pt, draw] at (1.5,3) (nand1) {};
    
    \draw (y.east) - ++(right:3mm) |- (nand1.input 1);
    \draw (y.east) - ++(right:3mm) |- (nand1.input 2);
    \draw (x) -- ([xshift=0.2cm]x) |- (nand2.input 1);
    \draw (nand1.output) -- node[above]{$\overline{Y}$} ([xshift=0.5cm]nand1.output) |- (nand2.input 2);
    
    \node[nand gate US, minimum size=32pt, draw, logic gate inputs=nn] at ($(z) + (6.5, 0.17)$) (nand3) {};
    
    \draw (z) -- ([xshift=0.2cm]z) |- (nand3.input 2);
    \draw (nand2.output) -- node[above]{$\overline{X.\overline{Y}}$} ([xshift=0.5cm]nand2.output) |- (nand3.input 1);
    
    \node[nand gate US, minimum size=32pt, draw, logic gate inputs=nn] at ($(w) + (9, 0.17)$) (nand4) {};
    
    \draw (w) -- ([xshift=0.2cm]w) |- (nand4.input 2);
    \draw (nand3.output) -- node[above]{$\overline{\overline{X.\overline{Y}}.Z}$} ([xshift=0.5cm]nand3.output) |- (nand4.input 1);
    
    \draw (nand4.output) -- node[above]{$\overline{\left(\overline{\overline{X.\overline{Y}}.Z}\right).W}$} ($(nand4) + (4, 0)$);
\end{tikzpicture}
\end{center}
\caption{Logic circuit using NAND gate}
\label{ckt1}
\end{figure}

\begin{table} [h!]
    \centering
    \begin{tabular}{ | c | c | c | c | c | }
    \hline
    X & Y & Z & W & Output \\
    \hline
    0 & 0 & 0 & 0 &  1\\
    0 & 0 & 0 & 1 &  0\\
    0 & 0 & 1 & 0 &  1\\
    0 & 0 & 1 & 1 &  1\\
    0 & 1 & 0 & 0 &  1\\
    0 & 1 & 0 & 1 &  0\\
    0 & 1 & 1 & 0 &  1\\
    0 & 1 & 1 & 1 &  1\\
    1 & 0 & 0 & 0 &  1\\
    1 & 0 & 0 & 1 &  0\\
    1 & 0 & 1 & 0 &  1\\
    1 & 0 & 1 & 1 &  0\\
    1 & 1 & 0 & 0 &  1\\
    1 & 1 & 0 & 1 &  0\\
    1 & 1 & 1 & 0 &  1\\
    1 & 1 & 1 & 1 &  1\\
     \hline
\end{tabular}
\caption{Truth table of above Logic circuit}
\label{Table1}
\end{table}
\end{document}
